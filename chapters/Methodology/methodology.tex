\section{Alternative Designs}
    \subsection{Needs Analysis}  % Brandon
% Needs Analysis is a formal, systematic process of identifying and evaluating training that should be done, or specific needs of an individual or group of employees, customers, suppliers, etc. Needs are often referred to as “gaps,” or the difference between what is currently done and what should be performed.
    \paragraph{}
    In order to develop a mobile application from start to finish, many variables must be considered in advance. We started off by determining what exact resources the Gordon Library supplies and which of them would be worth implementing into our application. Additionally, we thought of other resources perhaps outside of the library's scope that may aid our future users in their academic endeavours. From there, we found it would be best to devise a campus-wide survey with Google Forms that would provide us with honest results as to what of the library's resources people actually use currently and of those, which resources should be added to the mobile app.
    
    
    
    \paragraph{}
    From there, we moved to the "meat" of the project, which was the actual development of the mobile application. Between the three of us, we were most comfortable with coding in Java, a widely-used programming language that deals mostly with backend development or more simply, the code that lives "under the hood". Though, we still required a method at which to develop the frontend aspects, pertaining to what is actually seen on the screen. It was then up to us to learn JavaScript, a scripting language that enabled us to develop the frontend components. This would then allow us to fully develop our goal of a mobile application for the Gordon Library here at WPI.
    \paragraph{}
    In the background of our project's iterative process, we determined towards the very beginning of our three-term long project that the use of Jira and Confluence by Atlassian would be a great utility in aiding us in organizing our tasks and goals. None of us were necessarily "fluent" in either of the Atlassian software, but learned quite quickly as we progressed. Once a few of the initial weeks had passed, both Jira and Confluence were utilized for every aspect of the project. With this fluency, tasks were distributed evenly, project efficiency increased, and all the tasks could be tracked for later reference as needed in our results.
    
    \newpage
    \subsection{Functions (Specifications)} % Brandon
    % "What we are using to create the app and how"
    \paragraph{}
    Creating a mobile application requires a few items and even more when employed upon a group. These items include the following:
    \begin{itemize}
    
        \item IDE - Stands for Integrated Development Environment. An IDE is where a given programming language lives and the development of code exists. In our case, we used an IDE namely IntelliJ to house our project's code in which multiple programming languages and frameworks are used. JavaScript, 
        
        
        \item Programming Languages - Coding languages that allow for code to be written in. A programming language, when approaching a software-based project, is determined based on what is to be accomplished and in our case and some cases, what language each of the members of the team is mostly comfortable with. For the frontend development, and to allow for a nice layout and user interface for our mobile application, JavaScript, HTML, and Sass/SCSS were chosen to best permit us the capabilities we sought for our mobile app. The backend development languages chosen were Java, as it turned out that all three of us were most comfortable with Java, SQL, HQL, and JPQL. These four languages pertain and handle all the "under-the-hood" code. Thus buttons can have mapped functionalities, API consumers can be built and later displayed, etc. 
        
        
        \item Programming Frameworks - Frameworks are essentially reusable software environments which can be used for numerous different applications and enable one or many functionalities for the main program. Some frontend frameworks/libraries we used were Lit, Polymer, and Vaadin. These three frameworks/libraries not only provide us with additional functions for our primary program, they also simplify larger functions for implementation. Lit is a library that helps make the process of building web components in an fast and efficient manner. The Polymer library aid the process of creating custom elements far easier. Vaadin, a Java web application development framework, supports the creation and maintenance of web-based user interfaces of high quality more simplistic. In the case of backend frameworks, Spring, Hibernate ORM, Logback, and Lombok are used. Spring, being an open-source application framework, supports the development of Java-based applications. Hibernate ORM works in conjunction with SQL and handles the mapping of object-oriented domain models to a relational database. Logback is a logging framework that deals with data logging within Java. Lombok is a Java library that enhances the programming experience by automating certain trivial yet repetitive tasks.
    \end{itemize}



    \newpage
    \subsection{Conceptual Designs}  % Brandon
    % "Wireframes or similar?"
    
    \paragraph{}
    In the adolescent stages of our development process, numerous different designs for our mobile application were considered. These designs were primarily thought of and discussed in detail, however later on in our second term working on this IQP, we started to put our ideas to a concrete design with the use of the software Figma. Figma provides many capabilities to create conceptual designs that emulate a prototype or even final product. Though we used Figma for a short time, we were able to get a great sense as to the look and feel, or user interface, of what we want for our mobile application. From there, we determined it would be most efficient to put our thought of concepts and Figma models directly into code to produce our first prototype. From there, we hosted a team of testing volunteers to provide honest and constructive feedback of the application's user interface, user flow, and functionalities to revise our mobile app further towards the final product. 
    
    
    
    \newpage
    \subsection{Preliminary / Alternative Designs}  % Oliver
    
        \paragraph{}
        Not much is known about designing mobile academic library applications. Further, based on the limitations we knew from the start, our application would be unique; there is no precedent for what we are building. From the beginning of this project, we knew that common library application features would not be included in our application:
        \begin{itemize}
            \item Catalog search.
            \item Book availability \& checkout.
            \item E-books.
            \item Audio books.
        \end{itemize}
        
        \paragraph{}
        Nevertheless, we discussed many features that would be feasible to implement in our three-term timeframe:
        \begin{itemize}
            \item Library hours.
            \item Tech Suite booking.
            \item Occupancy count.
            \item News \& events.
            \item Library chat.
        \end{itemize}
        We may have not developed all of these features, as our survey has brought light upon the features sought after by students. More important, however, is the niche of features that prevalent in other mobile library applications, specifically, Tech Suite booking and occupancy count. No precedent exists for these features, so it was up to the team to not only program, but to also develop them. Develop, in the sense that we had to design before implement.
    
    \newpage
    \subsection{Feasibility Study} % Oliver
    % "The primary objective of this is to present a feasibility study scheme that enables system analysts to verify the preliminary essentials established by the idea innovator and to generate other genuine user’s requirements.” <How Feasible Analysis Is Essential For Mobile App Success?> - seems like this is integrated into your iterative process, right?"

        \subsubsection{Technical Feasibility}
        
            \paragraph{}
            By luck, all student team members were Computer Science majors with some form of history of software development. Some members had more experience than others, but everyone could bring something to the table. With our experience, we were familiar with the various tool for development and from the start, knew what we would use.
            
            \paragraph{}
            Our application requires a backend, meaning we needed some way of hosting our backend server. WPI has access to various servers as well as Amazon Web Services, so our hosting options were sufficient.
            
        \subsubsection{Legal Feasibility}
        
            \paragraph{}
            There is not much to say here. We, the student team members, operate under the guidelines set forth by WPI and our IQP advisor.
            
        \subsubsection{Economic Feasibility}
        
            \paragraph{}
            The application we aimed to build had no serious economic implications. At first, we thought we would need some paid tools, but eventually discovered alternatives. Quite frankly, the economics worked well for WPI. Given the developers were students, the work was free for the school. In fact, because students pay to attend, and therefore pay to participate in an IQP, which leads to a degree, there exists a mutual relationship beneficial to all.
            
        \subsubsection{Operational Feasibility}
        
            \paragraph{}
            WPI has offered Interactive Qualifying Projects for over 40 years, so there exists a strong foundation on which to operate projects such as ours. From a 9-page syllabus, to a 13-page report writing guidelines and templates, there was no reason to think that this project would be unfeasible from an operational perspective.
            
        \subsubsection{Scheduling Feasibility}
        
            \paragraph{}
            We were given three terms to complete this project. That seemed more than enough time to complete it. By luck, most student team members had flexible schedules and a lot of free now. We saw no trouble in meeting multiple times and week to complete this project within three terms.
    \newpage     
    \subsection{Modeling} % Oliver
    % "again, part of the iterative process you are talking about, right?"
    
        \paragraph{}
        We did not perform much modeling. We did complete some mock-ups and story boarding using a tool called "Figma". During the beginning is when our first survey results came back. The features that students wanted to see in our library application were so basic and minimal, that we felt it would be a waste of time formally designing a user interface. Rather, we were able to come up with an idea of the user interface through a single discussion.
    
    
    \newpage
    \subsection{Preliminary Findings / Outcomes} % John
    % "this would be in your work documentation (notebook), and should be summarized in the report as part of your results and discussion"

\newpage
\section{Design Verification / Validation} % John
    \newpage
    \subsection{Final Design Strategy}
    \paragraph{}
    In this section we will discuss our design strategy for producing our final product. The strategy was based around a set of steps we utilized in order to ensure we were working at a consistent and steady pace, on track to complete the final product by our deadline. Our first objective was to define what our goal with the development of this product would be: create an application which would provide users of the Gordon library with some sort of value. 
    \paragraph{}
    In order to move forward with the design, we established and agreed upon a broad goal. From there, we narrowed down this goal into sub goals. To do this, we asked ourselves what would we need to accomplish in order to achieve this primary objective. We established that we would need to produce something operational, usable, and relevant to the library goers. 
    \paragraph{}
    With these goals established, we brainstormed how would we accomplish each objective. In order to make an operational product, we would need to use development tools we were all familiar with. In order to make an app which was usable, we would need to see what potential library app users found attractive. The plan for this objective then, was to complete a prototype, have potential users test it, and adjust the UI in accordance to the feedback. Finally to complete the last objective, develop something relevant to the users, we concluded we must obtain information from potential users as for which library features they use and which they do not; for this, we concluded a school-wide survey would get us the information we needed. 
    \paragraph{}
    We now had three tasks in order to complete our broad goal: obtain information via a survey as for which library services are most relevant to potential users, develop prototypes and test them with potential application users to gauge which design respondents were most receptive too, and to develop a functional application based off the feedback from the prototype testing.  
    \paragraph{}
    The first objective to complete, as the other objectives depended on the results form this, was to write, deploy, and process the results of a survey relating to the library services the student body use, as well as weather or not they would use them if they were Incorporated into a mobile application.
    \paragraph{}
    Upon processing these results, we settled on which features we would include in the app and which we would not. From here, we developed mock ups as well as a prototype to see, at a sub-functional level, how this application would look when finalized. We tested this prototype on volunteers and made adjustments to the UI in accordance to the feedback. 
    \paragraph{}
    The final step was to produce a functional application based on the feedback from the previous two steps. At this stage in the development process, we knew exactly how this application would look and how it would function. 
    
    
    \newpage
    \subsection{Techniques for Deployment} % John
    % "Now that we know that loading is not as easy a we thought, this may just be a plan for what Jimmy could do after approvals, etc."
    \paragraph{}



%Pages: 8-12

% The Methodology Chapter:  Justifying the “What” and the “How” 
 
% The Methodology chapter is often the most challenging to write.  The chapter needs to state how you broke down your project into analytical components and the systematic process you used to achieve your goal.  Typically, the chapter is organized in terms of a set of project objectives that lead toward the goal.  Each objective typically involves answering certain research questions.  Research questions are answered by applying a methodology (research approach) and suitable methods (data collection techniques).  The types of questions asked will determine the appropriate research approach, and methods.  So, despite the name of the chapter, the methods aren’t really the most important part of this chapter!   

% The Best Methodology Chapters 
% •	Clarify the goal, objectives, research questions, and methods 
% •	Describe and analyze the limitations of the methodology/research 
% •	Describe and analyze the challenges to learning what you needed to learn (without drama or complaints) 
% •	Develop issues raised in the background section in a consistent and systematic fashion 
% This chapter should not read like a diary or timeline of what you did – rather, it should make clear what you set out to learn, why this information was important, and how you got it, why the methods you chose were a reasonable choice, how you overcame challenges and what limitations remain. The focus is on information: what you wanted to learn, and how that knowledge was subsequently used.   After you have clearly identified the what, discuss the how. Researchers can choose from, or combine, many methods to answer research questions and achieve objectives, so your job is not to present your approach as the only correct one. However, the reader should conclude that everything you did was intentional, purposeful, planned ahead of time, and based on good practices.   
% Content 
% •	Start with the purpose.  Begin by reminding readers of the goal of the project.  
% •	List the objectives in the introduction to the chapter, and use them to organize the chapter.  For many projects, it will make sense to have a section for each objective. 
% •	Begin each section with an explanation of the purpose of the objective.  This is not what you did but what information you intended to learn.  Readers will not be able to assess your methods unless they know what you are looking for.  Explain WHY you needed this information.  Make sure each purpose is clear before jumping into details of technique--the “how.” 
% •	Frame each objective in terms of a set of research questions.  These are the “big picture” questions you need to answer to satisfy that objective. 
% •	Describe HOW you got the information you needed.  Here is where the methods come in.As you plan the research, you may initially write up the methods that you will use to collect information, then change the tense as you report what you did do.   
% •	After describing the HOW, justify your choice of methodology and methods (another type of WHY). Drawing upon best practices for methods in the literature gives you more credibility as researchers. Provide enough, but not too much detail.  Someone should be able to repeat your work using information provided in this chapter and in associated Appendices. (For example, a general idea of the nature of interview questions should be evident in this chapter, but you could put the actual schedule of interview questions in an Appendix.)  
% •	Include a discussion of data analysis– that is, describe not only how you collected information, but also how you planned to use that information to achieve your goal. Wherever possible, think in terms of CRITERIA that you will use to find meaning in your data or to make decisions using data.  Sometimes lists or tables or charts are helpful for laying out criteria and judging options with respect to them. 
% •	Acknowledge and discuss problems, challenges, limitations, and flaws of your study.  This is important!  Your work will not be credible unless you recognize its limitations.  Proficient researchers describe the problems they had in gathering and analyzing information, and avoid overstated claims of validity.  Rather than judging your work to be successful, provide enough evidence for the reader to make the judgment. 
 
\textbf{% Structure 
% •	As in the Background chapter, provide a short introductory paragraph, positioned between the Methodology heading and the first section heading (for objective 1), providing a preview of the chapter for the reader.  Often an exact restatement of your Move 5 (from the overall Introduction) is a great introduction.  Number your objectives. 
% •	Here is a good sample introductory paragraph: 
% The goal of our project was to assess EGAT’s current environmental communication strategies and make recommendations for improvement specific to Mae Moh’s information needs.  In order to achieve this goal, we developed the following research objectives: 
% 1.	Build trust with EGAT employees and the Mae Moh communities to learn about their perspectives regarding EGAT’s impacts on local residents.  
% 2.	Identify EGAT’s communication strategies in terms of content, presentation and accessibility and identify villagers’ information needs regarding pollution and other environmental concerns. 
% 3.	Develop recommendations by comparing EGAT’s strategies with villagers’ needs to determine gaps.   
% In this chapter, we describe the methodology we developed to gather and analyze input from key stakeholders, and how the results of that analysis were drawn upon to develop recommendations for EGAT and the Mae Moh villagers. 
% •	The chapter should be organized by objective and research question, not method or chronology.  This chapter is not a diary of your project; don’t give a timeline of what you did, but rather establish what you set out to do and why.  This is a subtle issue of voice.  Following is an exaggerated example of the voice to avoid: “First we contacted key government officials who had knowledge about building regulations…. After speaking to government officials we met with representatives of advocacy groups for the hearing impaired…. Then we…”  See how that sounds like a diary?  Getting rid of the “time stamps” in this passage would help a lot.  Keep the focus on types of information you sought.  
% •	Use the past tense in the final IQP report to demonstrate what you have done.  . For example, “We will accomplish our goal by…” should be changed to “We accomplished our goal by….” 
% •	Avoid use of “We needed to interview..” or “It was necessary to find information on….” or “It was important to…”  There is no single way to do any project, so these are not very convincing claims.  Just say what you did. You can state things much more clearly in active voice: “We interviewed…in order to….”  “We sought information on…for the purpose of…” 
% •	Although the passive voice is common for writing scientific methodologies, we encourage you to write this chapter in the active voice.  For example, use “we interviewed three people” instead of “three people were interviewed.”  For more examples, consult a writing manual.   
 
% Example: The following segments from a Methodology chapter illustrate some of the key points described above. This example shows a strong “research voice.”  
 
% 3.1 Objective 1 
% Build trust with EGAT employees and Mae Moh communities to learn about their perspectives regarding EGAT’s impacts on local residents. 
% Studies show that outside researchers must overcome obstacles when becoming involved with and researching a culturally unfamiliar community.  Some even believe that outside researchers cannot understand or represent the experience of the community (Bridges, 2001).  Because of this… 
% As outsiders in Thailand, the biggest challenge that we faced was establishing trust and credibility with Mae Moh and EGAT. Gaining trust from the villagers was especially difficult because… 
% As outsiders with little credibility in the Mae Moh and EGAT communities, we adopted strategies recommended in our research for achieving open and trusting relationships with both.  These strategies required….  
% (There was much more to this objective.  The team went on to describe and defend their strategies for gaining trust with each of the stakeholders.) 
% 3.2 Objective 2  
% Identify EGAT’s communication strategies in terms of content and accessibility and identify Mae Moh’s information needs regarding pollution and other environmental concerns. 
% We set out to learn about EGAT’s current communication techniques and Mae Moh’s environmental information needs as a foundation for developing recommendations.  To evaluate communication in Mae Moh, we focused on two particularly important components of risk communication model development: informational content and accessibility.  Knowing these components from both perspectives allowed us to compare EGAT’s efforts with the community’s needs and determine disparities.  To determine the problems with content, and accessibility, we chose historical research and semi-standardized interviews due to limited information on interviewees.  The historical research mentioned earlier allowed us to discover previous efforts in environmental safety and communication.  Interviews at EGAT and informal discussion with communities helped to bridge the gaps in the literature.  Interview and discussion questions were based on the following research questions used to establish our information needs: 
% ▪	What informational content does EGAT communicate? 
% ▪	What do Mae Moh villagers want to know? 
% ▪	What is the presentation of information? 
% ▪	Who presents this information? 
% ▪	How does the community receive information? 
% ▪	How does the relationship between EGAT and the community influence reception? 
% Using our research questions, we formed interview and discussion questions specific to the interviewees’ responsibilities and knowledge. We were also careful to avoid offensive and unprofessional language.   
% We directed interview questions at EGAT employees.  In total, we interviewed 18 EGAT employees in several departments (see Table 1 for details.)  Each group member undertook a task during the interview process….   
% For interviewing community members, we faced similar language concerns.  A translator minimized such communication problems to the extent possible.  Our translator, Hatarat Poomkachar, is a social science researcher from Chulalongkorn University.  She facilitated communication between our team and the community.  She has had experience…We intended to present our research findings back to the community at the end of our research process to allow for corrections and feedback. However, due to time limitations and lack of a translator in our final week in Mae Moh, we could not implement this feedback process…. 
% With these difficulties in mind, we spoke with villagers to determine their understanding of EGAT’s operations and their awareness of communication efforts. In total, we spoke with 22 villagers, generally in a group discussion format (see Table 2 for details).  Personal interviews usually lasted about one hour while group discussions lasted longer, about 1.5-2 hours. We visited Pong Chai and Na Sak once, each for approximately 4-6 hours.  We visited Hua Fai twice due to villagers’ eagerness to speak to us and share their opinions.  We also spoke with teachers and students from the local high school, all of which are residents of villages in Mae Moh district.  We gathered general community knowledge of EGAT’s operations and pollution through informal discussion.  We discussed EGAT’s efforts in order to assess the public’s comprehension of and access to environmental information.  The data collected provided insight into possible communication improvements.  It indicated what informational content the people are concerned about, which presentation methods best suit their needs, and who they trust to communicate the information. 
 
% Features of Effective Findings Chapters 
% •	Move directly to stating claims, rather than rehashing methods all over again.  
% •	Avoid data dumping (present new knowledge, not raw data) 
% •	Contain in-depth analysis 
% •	Have firm but appropriately qualified claims rather than overstated claims  
% •	Refer and utilize the literature wherever possible 
% •	Are consistent with what you said you would do in the Methodology chapter}
