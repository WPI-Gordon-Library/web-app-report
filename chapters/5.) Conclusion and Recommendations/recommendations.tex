% Pages: 6-12


% The Conclusions and Recommendations Chapter 
% Often a reader will want to skim through this final chapter and be able to pick out quickly your specific conclusions and recommendations, so this chapter should be persuasive on its own.  
% Content:  Conclusions can be thought of as a summary of the key findings of your project.  In this sense, it may seem a bit repetitive from your Findings chapter. That’s OK, because sometimes people read only this chapter and not the Findings chapter.  This chapter, however, should be MUCH MORE SUCCINCT.  Instead of giving complete evidence to back up your findings, just suggest the nature of the evidence.  
% Almost all projects will also deliver Recommendations appropriate to your work. Make sure the nature of this chapter is consistent with what you said you would produce in your goal statement.  Generally, very little in this chapter should come as a complete surprise to the reader since findings have already been presented; recommendations should fall out naturally from that discussion.  Note that there are two types of recommendations, those made to address the problem you studied, and those made on next steps of study.  Depending on the nature of your project, you might emphasize recommendations more than conclusions or vice versa. 
% Structure 
% •	As with other chapters, start with a good introductory paragraph.  
% •	Often you can make this chapter more readable by using formatting like bold, italic, or underlining.  You have different options for organizing and formatting this chapter. Sometimes it can be effective to present conclusions and recommendations separately. In other cases, it may work well to present “pairs” of conclusions and recommendations. For example, you could say, 
% “Our results showed that…. In order to address this problem, we recommend that …”  
% •	In a similar manner as the Executive Summary and Findings chapter, make strategic use of formatted lists in this chapter so that readers can easily identify specific conclusions and recommendations. (See example given with Executive Summary.) 
% •	As always, choose your words VERY carefully. A common mistake for recommendations— both in writing and in presentations—is to use statements like “We feel that the Patent and Trademark Office should change its prior art search services by…”  No one should take an action based on how you feel. Rather, you should make clear that your conclusions and recommendations are based on evidence. Therefore, use wording like “We recommend that…” and then accompany that with evidence and justification: “Our results showed that …. Moreover, there was broad support for… Therefore, we conclude…” Keep an objective tone throughout and focus on developing persuasive arguments. Be careful with the word “should.” 
% •	Use active voice, and make clear the audience for each recommendation: “We recommend that Organization X…”  Use words that convey an appropriate level of confidence in your results and recommendations.   
% •	Point out any limitations of your project, and be careful not to make the scope of your claims more broad than is justified by your results. Your argument will be more persuasive if you acknowledge limitations and alternatives you considered. 
% •	Often it is difficult to figure out how to conclude this chapter, and a concept that may be helpful is to think of it as an inverse funnel. The whole report up to that point has been funneling down to the specific findings and the accomplishment of your goal, but then you want to funnel back out and point out things like the long-term implications of your work, who will benefit from your work, possible uses of your findings and recommendations by individuals or groups other than your project sponsor, and/or interesting questions that your work raises that could be pursued by others in the future. You’d like the reader to be thinking “WOW!” at the end of this chapter, which you can only accomplish with meaningful rather than superficial concluding remarks. 



\section{Development Approach}

\paragraph{}
To be clear, this section discusses the entirety of the development from preliminary


\section{Techniques for Deployment}

% TODO: Rewrite (from 3.10):
\paragraph{}
The end goal was to deploy the final application on mobile app stores: iOS and Android. When we began development, we shifted our approx from writing a native application to a web applica- tion. This choice removed the possibility of deployment to the the iOS app store, but seemed the most logical route to complete to project in time. Deployment was never achieved, and thus our recommendations to continue the project are discussed in our conclusion.