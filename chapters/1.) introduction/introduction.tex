

\paragraph{}
The overall use of handheld devices has increased dramatically within the past decade or so, especially in academic practices. As technology grows, our environment adapts to such changes. However, to keep up with the advancements of technology is no simple task and public services can fall short. The WPI George C. Gordon Library, for instance, offers high quality information resources, expert consultations and instructional services, and a variety of work spaces to support the WPI community. Students can utilize the library for its quiet study spaces, private group study rooms, research consultations, and more. On the technical side, the library staff is able to provide their services through a handful of different systems. These resources, on the other hand, are made available strictly on web-hosted applications, requiring a web browser. This is quite cumbersome and not user friendly on mobile devices whatsoever as a mobile website is merely a stripped-down version of the source web page. In our world, where convenience is key, the student demographic as turning away from using the library services.

\paragraph{}
As few steps had been taken towards improving the accessibility of such services, this project aimed to produce a mobile-first web application, centralizing many library services into one place to cater to mobile devices and promote convenience. We believed that by establishing a fully functioning mobile application, the ”traffic” to the library’s resources and services would increase, along with the foot traffic within the Gordon Library building.The expansion of other requested services, such as individual computer occupancy tracking so that users can know how many computers are available at a given time, and a remote book check-out service were beyond the scope of this IQP project.

\paragraph{}

    The Progressive Web Application (PWA) standard was utilized for the development of the mobile application. This means that our application meets the following criteria:
    
   \begin{itemize}[label={\checkmark}]
       \item Progressive  -- Works for most users on most devices and web browsers.
      \item Responsive   -- User interface (UI) fits any common form factor: desktop, table, mobile.
     \item App-like     -- Feels and acts like a native mobile app on mobile devices.
        \item Safe         -- Content served via encrypted HTTPS to secure user accounts and interactions.
        \item Discoverable -- Identifiable as a website, found on search engines.
        \item Installable  -- Able to be added to a mobile device's homescreen as a normal app.
        \item Linkable     -- Easily shared via a URL and does not require complex installation.
    \end{itemize}

\paragraph{}

    The PWA standard is still being adopted by some browsers and devices. At the time of the \appname app development, support for every feature may not be available to all users, depending on device and browser.
