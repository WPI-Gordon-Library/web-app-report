\section{Academic Library Features}

    \noindent{} The following is based from the research gathered by a study conducted by Mansouri and  Soleymani in 2018. Each resource was considered based on the capabilities they provide and if they fit within the demographic of our mobile library application..
    
    \paragraph{Search (by barcode/QR code scan)}
    We assumed these features refer to a user's ability to search the library catalog. We decided to not implement this feature in the \appname app. However, it is an important one, as according to this study, all existing library apps reviewed had this feature. We believe for a mobile app that is intended to be a replacement for a desktop website to be successful, it must provide the most common services provided by its desktop website counterpart.
    
    \paragraph{Tutorial}
    We assume this feature refers to a visual tutorial provided within the mobile app that teaches new users how to use said mobile app. The study found that 70\% of library mobile apps had this feature. A tutorial allows new users to get a better understanding of how to use a platform. Without a tutorial, many new users may be lost, therefore rendering the mobile app useless for them. Although we considered adding a tutorial feature to our app, it was not included in our final design.
    
    \paragraph{Ask a librarian}
    We assumed this feature refers to a user's ability to submit questions to a librarian. The study found all library mobile apps included this feature. Our design plans included a similar feature: the library chat.
    
    \paragraph{Hours}
    Straight-forward features that provides users with library hours. This feature is part of the \appname app design plan.
    
    \paragraph{Events}
    We assumed this feature refers to a user's ability to view library events and/or a calendar. The study found that 90\% of library mobile apps have this feature. Implementation of this feature is in our design plan. Users should not have to use a separate service.
    
    \paragraph{FAQ/Feedback}
    It surprised us that only 30\% of library mobile apps have this feature. The whole point of a library application is to provide students easier access to the WPI library's various services. It is impossible to make a perfect product, because there is always some disconnect between what the developers think consumers want, and what the consumers actually want. The design plan includes a platform for users of the \appname app to provide feedback on their experience.