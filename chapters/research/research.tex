\paragraph{}
\TODO Intro paragraph.
\paragraph{}
\section{Peer-reviewed articles findings}
\subsection{Why a library app is important}
\subsection{What makes an app effective}
\subsection{What features would audiences be receptive too}
\subsection{What design choices would audiences be receptive too}
\subsection{What impact would an app like this have on the community}

\section{Survey}

\section{Past IQPs}

\section{Mobile Apps}
\subsection{Harvard}
\paragraph{}
Harvard's library app, HarvardLibrary, featured a minimalist home screen with the school logo in the upper-right hand corner, a help button in the lower-left, a 'login' feature in the lower right, and a circular 'start' button in the center. The screen featured only three colors; white, black and crimson red. Clicking the 'start' button pulled up the same login screen that clicking the 'login' button pulled up.  Unfortunately, we were not able to further explore this mobile app because we do not have account affiliated with the University. 
\subsection{Salisbury}
\paragraph{}
Sailsbury Universitie's Library app, SU Libraries, featured a noticeably less minimalist design than Harvard's, and had a more unrefined look.  Fortunately, we were able to access the applications features, however. The home screen included a search bar, and nine features: "Library Hours", "Research Help", "Room Reservations", "Self Checkout", "SU Libraries MakerLab", "Device Availability", "Building Maps", "Helpful Links" and "Contact Information". It also contained a bar at the bottom with five navigation options: "Home", "Library News", "Chat", "My Card" and "About". Additionally, there was a link to the libraries social media accounts.
\paragraph{}
The "Library Hours" feature lead to a screen with minimal interactivity. The screen contained the opening and closing times for all days from the current day to approximately three months in advance. The "Research Help" feature took the user to a page with a somewhat similar layout; a page in which you could scroll downward, containing the subjects in which you could get research help. This page, however, was interactive. The user could click on each subject option, taking them to a page with information about who to contact, and how to contact them, if they are seeking research help in that specific area.  The "Study Room Reservations" feature as well as the "Devise Availability" feature has a similar layout to these two.
\paragraph{}
The "Building Maps" feature lead to a page containing the layout of each floor of the building. The only interactivity on this page was the option to click on any of the diagrams for an enlarged view of the diagram. 
\paragraph{}
The "SU Libraries MakerLab" feature took the user to a page which seemed functioned like an app in and of itself. The page it took the user to contained a variety of features relating to the Maker Lab. Such as a feature displaying the time its open for today as well as a feature displaying which days it is open. It also had a feature containing taking the user to a page with the Maker Lab policies, a feature allowing users to make an appointment in the maker space, and finally the contact information for the maker space. 
\paragraph{}

\subsection{University of Dallas}
\subsection{University of Sydney}