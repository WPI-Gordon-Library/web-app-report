\paragraph{}

    The WPI George C. Gordon Library offers high quality information resources, expert consultations, and a variety of meeting work spaces to support the WPI community. Students can utilize the library for its quiet study spaces, private rooms, research consultations, and more. On a technical side, the library staff is able to provide their services through a handful of different systems. With services hosted on multiple platforms, students were responsible for familiarizing themselves with them all. This was the problem we aimed to solve. In a world where convenience is key, this project and its \appname app aimed to produce a mobile-first web application, centralizing many library services into one place.

\paragraph{}

    The \appname app was developed as a Progressive Web Application (PWA). This means that our application meets the following criteria:
    
    \begin{itemize}[label={\checkmark}]
        \item Progressive  -- Works for most users on most devices and web browsers.
        \item Responsive   -- User interface (UI) fits any common form factor: desktop, table, mobile.
        \item App-like     -- Feels and acts like a native mobile app on mobile devices.
        \item Safe         -- Content served via encrypted HTTPS to secure user accounts and interactions.
        \item Discoverable -- Identifiable as a website, found on search engines.
        \item Installable  -- Able to be added to a mobile device's homescreen as a normal app.
        \item Linkable     -- Easily shared via a URL and does not require complex installation.
    \end{itemize}

\paragraph{}

    The PWA standard is still being adopted by some browsers and devices. At the time of the \appname app release, support for every feature may not be available to all users, depending on device and browser.
